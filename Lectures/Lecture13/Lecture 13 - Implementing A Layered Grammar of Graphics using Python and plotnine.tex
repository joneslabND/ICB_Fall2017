
% Default to the notebook output style

    


% Inherit from the specified cell style.




    
\documentclass[11pt]{article}

    
    
    \usepackage[T1]{fontenc}
    % Nicer default font (+ math font) than Computer Modern for most use cases
    \usepackage{mathpazo}

    % Basic figure setup, for now with no caption control since it's done
    % automatically by Pandoc (which extracts ![](path) syntax from Markdown).
    \usepackage{graphicx}
    % We will generate all images so they have a width \maxwidth. This means
    % that they will get their normal width if they fit onto the page, but
    % are scaled down if they would overflow the margins.
    \makeatletter
    \def\maxwidth{\ifdim\Gin@nat@width>\linewidth\linewidth
    \else\Gin@nat@width\fi}
    \makeatother
    \let\Oldincludegraphics\includegraphics
    % Set max figure width to be 80% of text width, for now hardcoded.
    \renewcommand{\includegraphics}[1]{\Oldincludegraphics[width=.8\maxwidth]{#1}}
    % Ensure that by default, figures have no caption (until we provide a
    % proper Figure object with a Caption API and a way to capture that
    % in the conversion process - todo).
    \usepackage{caption}
    \DeclareCaptionLabelFormat{nolabel}{}
    \captionsetup{labelformat=nolabel}

    \usepackage{adjustbox} % Used to constrain images to a maximum size 
    \usepackage{xcolor} % Allow colors to be defined
    \usepackage{enumerate} % Needed for markdown enumerations to work
    \usepackage{geometry} % Used to adjust the document margins
    \usepackage{amsmath} % Equations
    \usepackage{amssymb} % Equations
    \usepackage{textcomp} % defines textquotesingle
    % Hack from http://tex.stackexchange.com/a/47451/13684:
    \AtBeginDocument{%
        \def\PYZsq{\textquotesingle}% Upright quotes in Pygmentized code
    }
    \usepackage{upquote} % Upright quotes for verbatim code
    \usepackage{eurosym} % defines \euro
    \usepackage[mathletters]{ucs} % Extended unicode (utf-8) support
    \usepackage[utf8x]{inputenc} % Allow utf-8 characters in the tex document
    \usepackage{fancyvrb} % verbatim replacement that allows latex
    \usepackage{grffile} % extends the file name processing of package graphics 
                         % to support a larger range 
    % The hyperref package gives us a pdf with properly built
    % internal navigation ('pdf bookmarks' for the table of contents,
    % internal cross-reference links, web links for URLs, etc.)
    \usepackage{hyperref}
    \usepackage{longtable} % longtable support required by pandoc >1.10
    \usepackage{booktabs}  % table support for pandoc > 1.12.2
    \usepackage[inline]{enumitem} % IRkernel/repr support (it uses the enumerate* environment)
    \usepackage[normalem]{ulem} % ulem is needed to support strikethroughs (\sout)
                                % normalem makes italics be italics, not underlines
    

    
    
    % Colors for the hyperref package
    \definecolor{urlcolor}{rgb}{0,.145,.698}
    \definecolor{linkcolor}{rgb}{.71,0.21,0.01}
    \definecolor{citecolor}{rgb}{.12,.54,.11}

    % ANSI colors
    \definecolor{ansi-black}{HTML}{3E424D}
    \definecolor{ansi-black-intense}{HTML}{282C36}
    \definecolor{ansi-red}{HTML}{E75C58}
    \definecolor{ansi-red-intense}{HTML}{B22B31}
    \definecolor{ansi-green}{HTML}{00A250}
    \definecolor{ansi-green-intense}{HTML}{007427}
    \definecolor{ansi-yellow}{HTML}{DDB62B}
    \definecolor{ansi-yellow-intense}{HTML}{B27D12}
    \definecolor{ansi-blue}{HTML}{208FFB}
    \definecolor{ansi-blue-intense}{HTML}{0065CA}
    \definecolor{ansi-magenta}{HTML}{D160C4}
    \definecolor{ansi-magenta-intense}{HTML}{A03196}
    \definecolor{ansi-cyan}{HTML}{60C6C8}
    \definecolor{ansi-cyan-intense}{HTML}{258F8F}
    \definecolor{ansi-white}{HTML}{C5C1B4}
    \definecolor{ansi-white-intense}{HTML}{A1A6B2}

    % commands and environments needed by pandoc snippets
    % extracted from the output of `pandoc -s`
    \providecommand{\tightlist}{%
      \setlength{\itemsep}{0pt}\setlength{\parskip}{0pt}}
    \DefineVerbatimEnvironment{Highlighting}{Verbatim}{commandchars=\\\{\}}
    % Add ',fontsize=\small' for more characters per line
    \newenvironment{Shaded}{}{}
    \newcommand{\KeywordTok}[1]{\textcolor[rgb]{0.00,0.44,0.13}{\textbf{{#1}}}}
    \newcommand{\DataTypeTok}[1]{\textcolor[rgb]{0.56,0.13,0.00}{{#1}}}
    \newcommand{\DecValTok}[1]{\textcolor[rgb]{0.25,0.63,0.44}{{#1}}}
    \newcommand{\BaseNTok}[1]{\textcolor[rgb]{0.25,0.63,0.44}{{#1}}}
    \newcommand{\FloatTok}[1]{\textcolor[rgb]{0.25,0.63,0.44}{{#1}}}
    \newcommand{\CharTok}[1]{\textcolor[rgb]{0.25,0.44,0.63}{{#1}}}
    \newcommand{\StringTok}[1]{\textcolor[rgb]{0.25,0.44,0.63}{{#1}}}
    \newcommand{\CommentTok}[1]{\textcolor[rgb]{0.38,0.63,0.69}{\textit{{#1}}}}
    \newcommand{\OtherTok}[1]{\textcolor[rgb]{0.00,0.44,0.13}{{#1}}}
    \newcommand{\AlertTok}[1]{\textcolor[rgb]{1.00,0.00,0.00}{\textbf{{#1}}}}
    \newcommand{\FunctionTok}[1]{\textcolor[rgb]{0.02,0.16,0.49}{{#1}}}
    \newcommand{\RegionMarkerTok}[1]{{#1}}
    \newcommand{\ErrorTok}[1]{\textcolor[rgb]{1.00,0.00,0.00}{\textbf{{#1}}}}
    \newcommand{\NormalTok}[1]{{#1}}
    
    % Additional commands for more recent versions of Pandoc
    \newcommand{\ConstantTok}[1]{\textcolor[rgb]{0.53,0.00,0.00}{{#1}}}
    \newcommand{\SpecialCharTok}[1]{\textcolor[rgb]{0.25,0.44,0.63}{{#1}}}
    \newcommand{\VerbatimStringTok}[1]{\textcolor[rgb]{0.25,0.44,0.63}{{#1}}}
    \newcommand{\SpecialStringTok}[1]{\textcolor[rgb]{0.73,0.40,0.53}{{#1}}}
    \newcommand{\ImportTok}[1]{{#1}}
    \newcommand{\DocumentationTok}[1]{\textcolor[rgb]{0.73,0.13,0.13}{\textit{{#1}}}}
    \newcommand{\AnnotationTok}[1]{\textcolor[rgb]{0.38,0.63,0.69}{\textbf{\textit{{#1}}}}}
    \newcommand{\CommentVarTok}[1]{\textcolor[rgb]{0.38,0.63,0.69}{\textbf{\textit{{#1}}}}}
    \newcommand{\VariableTok}[1]{\textcolor[rgb]{0.10,0.09,0.49}{{#1}}}
    \newcommand{\ControlFlowTok}[1]{\textcolor[rgb]{0.00,0.44,0.13}{\textbf{{#1}}}}
    \newcommand{\OperatorTok}[1]{\textcolor[rgb]{0.40,0.40,0.40}{{#1}}}
    \newcommand{\BuiltInTok}[1]{{#1}}
    \newcommand{\ExtensionTok}[1]{{#1}}
    \newcommand{\PreprocessorTok}[1]{\textcolor[rgb]{0.74,0.48,0.00}{{#1}}}
    \newcommand{\AttributeTok}[1]{\textcolor[rgb]{0.49,0.56,0.16}{{#1}}}
    \newcommand{\InformationTok}[1]{\textcolor[rgb]{0.38,0.63,0.69}{\textbf{\textit{{#1}}}}}
    \newcommand{\WarningTok}[1]{\textcolor[rgb]{0.38,0.63,0.69}{\textbf{\textit{{#1}}}}}
    
    
    % Define a nice break command that doesn't care if a line doesn't already
    % exist.
    \def\br{\hspace*{\fill} \\* }
    % Math Jax compatability definitions
    \def\gt{>}
    \def\lt{<}
    
    
    

    % Pygments definitions
    
\makeatletter
\def\PY@reset{\let\PY@it=\relax \let\PY@bf=\relax%
    \let\PY@ul=\relax \let\PY@tc=\relax%
    \let\PY@bc=\relax \let\PY@ff=\relax}
\def\PY@tok#1{\csname PY@tok@#1\endcsname}
\def\PY@toks#1+{\ifx\relax#1\empty\else%
    \PY@tok{#1}\expandafter\PY@toks\fi}
\def\PY@do#1{\PY@bc{\PY@tc{\PY@ul{%
    \PY@it{\PY@bf{\PY@ff{#1}}}}}}}
\def\PY#1#2{\PY@reset\PY@toks#1+\relax+\PY@do{#2}}

\expandafter\def\csname PY@tok@gd\endcsname{\def\PY@tc##1{\textcolor[rgb]{0.63,0.00,0.00}{##1}}}
\expandafter\def\csname PY@tok@gu\endcsname{\let\PY@bf=\textbf\def\PY@tc##1{\textcolor[rgb]{0.50,0.00,0.50}{##1}}}
\expandafter\def\csname PY@tok@gt\endcsname{\def\PY@tc##1{\textcolor[rgb]{0.00,0.27,0.87}{##1}}}
\expandafter\def\csname PY@tok@gs\endcsname{\let\PY@bf=\textbf}
\expandafter\def\csname PY@tok@gr\endcsname{\def\PY@tc##1{\textcolor[rgb]{1.00,0.00,0.00}{##1}}}
\expandafter\def\csname PY@tok@cm\endcsname{\let\PY@it=\textit\def\PY@tc##1{\textcolor[rgb]{0.25,0.50,0.50}{##1}}}
\expandafter\def\csname PY@tok@vg\endcsname{\def\PY@tc##1{\textcolor[rgb]{0.10,0.09,0.49}{##1}}}
\expandafter\def\csname PY@tok@vi\endcsname{\def\PY@tc##1{\textcolor[rgb]{0.10,0.09,0.49}{##1}}}
\expandafter\def\csname PY@tok@vm\endcsname{\def\PY@tc##1{\textcolor[rgb]{0.10,0.09,0.49}{##1}}}
\expandafter\def\csname PY@tok@mh\endcsname{\def\PY@tc##1{\textcolor[rgb]{0.40,0.40,0.40}{##1}}}
\expandafter\def\csname PY@tok@cs\endcsname{\let\PY@it=\textit\def\PY@tc##1{\textcolor[rgb]{0.25,0.50,0.50}{##1}}}
\expandafter\def\csname PY@tok@ge\endcsname{\let\PY@it=\textit}
\expandafter\def\csname PY@tok@vc\endcsname{\def\PY@tc##1{\textcolor[rgb]{0.10,0.09,0.49}{##1}}}
\expandafter\def\csname PY@tok@il\endcsname{\def\PY@tc##1{\textcolor[rgb]{0.40,0.40,0.40}{##1}}}
\expandafter\def\csname PY@tok@go\endcsname{\def\PY@tc##1{\textcolor[rgb]{0.53,0.53,0.53}{##1}}}
\expandafter\def\csname PY@tok@cp\endcsname{\def\PY@tc##1{\textcolor[rgb]{0.74,0.48,0.00}{##1}}}
\expandafter\def\csname PY@tok@gi\endcsname{\def\PY@tc##1{\textcolor[rgb]{0.00,0.63,0.00}{##1}}}
\expandafter\def\csname PY@tok@gh\endcsname{\let\PY@bf=\textbf\def\PY@tc##1{\textcolor[rgb]{0.00,0.00,0.50}{##1}}}
\expandafter\def\csname PY@tok@ni\endcsname{\let\PY@bf=\textbf\def\PY@tc##1{\textcolor[rgb]{0.60,0.60,0.60}{##1}}}
\expandafter\def\csname PY@tok@nl\endcsname{\def\PY@tc##1{\textcolor[rgb]{0.63,0.63,0.00}{##1}}}
\expandafter\def\csname PY@tok@nn\endcsname{\let\PY@bf=\textbf\def\PY@tc##1{\textcolor[rgb]{0.00,0.00,1.00}{##1}}}
\expandafter\def\csname PY@tok@no\endcsname{\def\PY@tc##1{\textcolor[rgb]{0.53,0.00,0.00}{##1}}}
\expandafter\def\csname PY@tok@na\endcsname{\def\PY@tc##1{\textcolor[rgb]{0.49,0.56,0.16}{##1}}}
\expandafter\def\csname PY@tok@nb\endcsname{\def\PY@tc##1{\textcolor[rgb]{0.00,0.50,0.00}{##1}}}
\expandafter\def\csname PY@tok@nc\endcsname{\let\PY@bf=\textbf\def\PY@tc##1{\textcolor[rgb]{0.00,0.00,1.00}{##1}}}
\expandafter\def\csname PY@tok@nd\endcsname{\def\PY@tc##1{\textcolor[rgb]{0.67,0.13,1.00}{##1}}}
\expandafter\def\csname PY@tok@ne\endcsname{\let\PY@bf=\textbf\def\PY@tc##1{\textcolor[rgb]{0.82,0.25,0.23}{##1}}}
\expandafter\def\csname PY@tok@nf\endcsname{\def\PY@tc##1{\textcolor[rgb]{0.00,0.00,1.00}{##1}}}
\expandafter\def\csname PY@tok@si\endcsname{\let\PY@bf=\textbf\def\PY@tc##1{\textcolor[rgb]{0.73,0.40,0.53}{##1}}}
\expandafter\def\csname PY@tok@s2\endcsname{\def\PY@tc##1{\textcolor[rgb]{0.73,0.13,0.13}{##1}}}
\expandafter\def\csname PY@tok@nt\endcsname{\let\PY@bf=\textbf\def\PY@tc##1{\textcolor[rgb]{0.00,0.50,0.00}{##1}}}
\expandafter\def\csname PY@tok@nv\endcsname{\def\PY@tc##1{\textcolor[rgb]{0.10,0.09,0.49}{##1}}}
\expandafter\def\csname PY@tok@s1\endcsname{\def\PY@tc##1{\textcolor[rgb]{0.73,0.13,0.13}{##1}}}
\expandafter\def\csname PY@tok@dl\endcsname{\def\PY@tc##1{\textcolor[rgb]{0.73,0.13,0.13}{##1}}}
\expandafter\def\csname PY@tok@ch\endcsname{\let\PY@it=\textit\def\PY@tc##1{\textcolor[rgb]{0.25,0.50,0.50}{##1}}}
\expandafter\def\csname PY@tok@m\endcsname{\def\PY@tc##1{\textcolor[rgb]{0.40,0.40,0.40}{##1}}}
\expandafter\def\csname PY@tok@gp\endcsname{\let\PY@bf=\textbf\def\PY@tc##1{\textcolor[rgb]{0.00,0.00,0.50}{##1}}}
\expandafter\def\csname PY@tok@sh\endcsname{\def\PY@tc##1{\textcolor[rgb]{0.73,0.13,0.13}{##1}}}
\expandafter\def\csname PY@tok@ow\endcsname{\let\PY@bf=\textbf\def\PY@tc##1{\textcolor[rgb]{0.67,0.13,1.00}{##1}}}
\expandafter\def\csname PY@tok@sx\endcsname{\def\PY@tc##1{\textcolor[rgb]{0.00,0.50,0.00}{##1}}}
\expandafter\def\csname PY@tok@bp\endcsname{\def\PY@tc##1{\textcolor[rgb]{0.00,0.50,0.00}{##1}}}
\expandafter\def\csname PY@tok@c1\endcsname{\let\PY@it=\textit\def\PY@tc##1{\textcolor[rgb]{0.25,0.50,0.50}{##1}}}
\expandafter\def\csname PY@tok@fm\endcsname{\def\PY@tc##1{\textcolor[rgb]{0.00,0.00,1.00}{##1}}}
\expandafter\def\csname PY@tok@o\endcsname{\def\PY@tc##1{\textcolor[rgb]{0.40,0.40,0.40}{##1}}}
\expandafter\def\csname PY@tok@kc\endcsname{\let\PY@bf=\textbf\def\PY@tc##1{\textcolor[rgb]{0.00,0.50,0.00}{##1}}}
\expandafter\def\csname PY@tok@c\endcsname{\let\PY@it=\textit\def\PY@tc##1{\textcolor[rgb]{0.25,0.50,0.50}{##1}}}
\expandafter\def\csname PY@tok@mf\endcsname{\def\PY@tc##1{\textcolor[rgb]{0.40,0.40,0.40}{##1}}}
\expandafter\def\csname PY@tok@err\endcsname{\def\PY@bc##1{\setlength{\fboxsep}{0pt}\fcolorbox[rgb]{1.00,0.00,0.00}{1,1,1}{\strut ##1}}}
\expandafter\def\csname PY@tok@mb\endcsname{\def\PY@tc##1{\textcolor[rgb]{0.40,0.40,0.40}{##1}}}
\expandafter\def\csname PY@tok@ss\endcsname{\def\PY@tc##1{\textcolor[rgb]{0.10,0.09,0.49}{##1}}}
\expandafter\def\csname PY@tok@sr\endcsname{\def\PY@tc##1{\textcolor[rgb]{0.73,0.40,0.53}{##1}}}
\expandafter\def\csname PY@tok@mo\endcsname{\def\PY@tc##1{\textcolor[rgb]{0.40,0.40,0.40}{##1}}}
\expandafter\def\csname PY@tok@kd\endcsname{\let\PY@bf=\textbf\def\PY@tc##1{\textcolor[rgb]{0.00,0.50,0.00}{##1}}}
\expandafter\def\csname PY@tok@mi\endcsname{\def\PY@tc##1{\textcolor[rgb]{0.40,0.40,0.40}{##1}}}
\expandafter\def\csname PY@tok@kn\endcsname{\let\PY@bf=\textbf\def\PY@tc##1{\textcolor[rgb]{0.00,0.50,0.00}{##1}}}
\expandafter\def\csname PY@tok@cpf\endcsname{\let\PY@it=\textit\def\PY@tc##1{\textcolor[rgb]{0.25,0.50,0.50}{##1}}}
\expandafter\def\csname PY@tok@kr\endcsname{\let\PY@bf=\textbf\def\PY@tc##1{\textcolor[rgb]{0.00,0.50,0.00}{##1}}}
\expandafter\def\csname PY@tok@s\endcsname{\def\PY@tc##1{\textcolor[rgb]{0.73,0.13,0.13}{##1}}}
\expandafter\def\csname PY@tok@kp\endcsname{\def\PY@tc##1{\textcolor[rgb]{0.00,0.50,0.00}{##1}}}
\expandafter\def\csname PY@tok@w\endcsname{\def\PY@tc##1{\textcolor[rgb]{0.73,0.73,0.73}{##1}}}
\expandafter\def\csname PY@tok@kt\endcsname{\def\PY@tc##1{\textcolor[rgb]{0.69,0.00,0.25}{##1}}}
\expandafter\def\csname PY@tok@sc\endcsname{\def\PY@tc##1{\textcolor[rgb]{0.73,0.13,0.13}{##1}}}
\expandafter\def\csname PY@tok@sb\endcsname{\def\PY@tc##1{\textcolor[rgb]{0.73,0.13,0.13}{##1}}}
\expandafter\def\csname PY@tok@sa\endcsname{\def\PY@tc##1{\textcolor[rgb]{0.73,0.13,0.13}{##1}}}
\expandafter\def\csname PY@tok@k\endcsname{\let\PY@bf=\textbf\def\PY@tc##1{\textcolor[rgb]{0.00,0.50,0.00}{##1}}}
\expandafter\def\csname PY@tok@se\endcsname{\let\PY@bf=\textbf\def\PY@tc##1{\textcolor[rgb]{0.73,0.40,0.13}{##1}}}
\expandafter\def\csname PY@tok@sd\endcsname{\let\PY@it=\textit\def\PY@tc##1{\textcolor[rgb]{0.73,0.13,0.13}{##1}}}

\def\PYZbs{\char`\\}
\def\PYZus{\char`\_}
\def\PYZob{\char`\{}
\def\PYZcb{\char`\}}
\def\PYZca{\char`\^}
\def\PYZam{\char`\&}
\def\PYZlt{\char`\<}
\def\PYZgt{\char`\>}
\def\PYZsh{\char`\#}
\def\PYZpc{\char`\%}
\def\PYZdl{\char`\$}
\def\PYZhy{\char`\-}
\def\PYZsq{\char`\'}
\def\PYZdq{\char`\"}
\def\PYZti{\char`\~}
% for compatibility with earlier versions
\def\PYZat{@}
\def\PYZlb{[}
\def\PYZrb{]}
\makeatother


    % Exact colors from NB
    \definecolor{incolor}{rgb}{0.0, 0.0, 0.5}
    \definecolor{outcolor}{rgb}{0.545, 0.0, 0.0}



    
    % Prevent overflowing lines due to hard-to-break entities
    \sloppy 
    % Setup hyperref package
    \hypersetup{
      breaklinks=true,  % so long urls are correctly broken across lines
      colorlinks=true,
      urlcolor=urlcolor,
      linkcolor=linkcolor,
      citecolor=citecolor,
      }
    % Slightly bigger margins than the latex defaults
    
    \geometry{verbose,tmargin=1in,bmargin=1in,lmargin=1in,rmargin=1in}
    
    

    \begin{document}
    
    
    \section*{Lecture 13 - Implementing A Layered Grammar of Graphics using Python and plotnine}
    

    
    \subsection*{Tutorial}\label{tutorial}

Most of today will be spent working through examples of a variety of
plot types that can be generated using R and ggplot2. Work through the
provided code and talk to your neighbor about what is happening and why
it works. Make sure to ask any questions you may have.

~

A lot of the syntax in plotnine is very similar to that of ggplot2. The
ggplot cheatsheet is therefore useful, but not exactly correct in terms
of syntax. The following website has a complete documentation of
plotnine and will help to translate ggplot2 syntax into plotnine syntax:
http://plotnine.readthedocs.io/en/stable/api.html.

~

    \begin{Verbatim}[commandchars=\\\{\}]
{\color{incolor}In [{\color{incolor}1}]:} \PY{k+kn}{import} \PY{n+nn}{numpy}
        \PY{k+kn}{import} \PY{n+nn}{pandas}
        \PY{k+kn}{from} \PY{n+nn}{plotnine} \PY{k+kn}{import} \PY{o}{*}
        
        \PY{n}{mpg}\PY{o}{=}\PY{n}{pandas}\PY{o}{.}\PY{n}{read\PYZus{}csv}\PY{p}{(}\PY{l+s+s2}{\PYZdq{}}\PY{l+s+s2}{mpg.txt}\PY{l+s+s2}{\PYZdq{}}\PY{p}{,}\PY{n}{sep}\PY{o}{=}\PY{l+s+s2}{\PYZdq{}}\PY{l+s+se}{\PYZbs{}t}\PY{l+s+s2}{\PYZdq{}}\PY{p}{,}\PY{n}{header}\PY{o}{=}\PY{l+m+mi}{0}\PY{p}{)}
        
        \PY{n}{mpg}\PY{o}{.}\PY{n}{shape}
        
        \PY{n}{mpg}\PY{o}{.}\PY{n}{head}\PY{p}{(}\PY{l+m+mi}{5}\PY{p}{)}
\end{Verbatim}

    \begin{Verbatim}[commandchars=\\\{\}]
/anaconda/lib/python2.7/site-packages/statsmodels/compat/pandas.py:56: FutureWarning: The pandas.core.datetools module is deprecated and will be removed in a future version. Please use the pandas.tseries module instead.
  from pandas.core import datetools

    \end{Verbatim}

            \begin{Verbatim}[commandchars=\\\{\}]
{\color{outcolor}Out[{\color{outcolor}1}]:}   manufacturer model  displ  year  cyl       trans drv  cty  hwy
        0         audi    a4    1.8  1999    4    auto(l5)   f   18   29
        1         audi    a4    1.8  1999    4  manual(m5)   f   21   29
        2         audi    a4    2.0  2008    4  manual(m6)   f   20   31
        3         audi    a4    2.0  2008    4    auto(av)   f   21   30
        4         audi    a4    2.8  1999    6    auto(l5)   f   16   26
\end{Verbatim}
        
    \begin{Verbatim}[commandchars=\\\{\}]
{\color{incolor}In [{\color{incolor}2}]:} \PY{c+c1}{\PYZsh{} plot of displacement (engine size) vs. city miles per gallon (cty)}
        \PY{n}{a}\PY{o}{=}\PY{n}{ggplot}\PY{p}{(}\PY{n}{mpg}\PY{p}{,}\PY{n}{aes}\PY{p}{(}\PY{n}{x}\PY{o}{=}\PY{l+s+s2}{\PYZdq{}}\PY{l+s+s2}{displ}\PY{l+s+s2}{\PYZdq{}}\PY{p}{,}\PY{n}{y}\PY{o}{=}\PY{l+s+s2}{\PYZdq{}}\PY{l+s+s2}{cty}\PY{l+s+s2}{\PYZdq{}}\PY{p}{)}\PY{p}{)}
        \PY{n}{a}\PY{o}{+}\PY{n}{geom\PYZus{}point}\PY{p}{(}\PY{p}{)}\PY{o}{+}\PY{n}{coord\PYZus{}cartesian}\PY{p}{(}\PY{p}{)}
\end{Verbatim}

    \begin{center}
    \adjustimage{max size={0.9\linewidth}{0.9\paperheight}}{Lecture 13 - Implementing A Layered Grammar of Graphics using Python and plotnine_files/Lecture 13 - Implementing A Layered Grammar of Graphics using Python and plotnine_2_0.png}
    \end{center}
    { \hspace*{\fill} \\}
    
            \begin{Verbatim}[commandchars=\\\{\}]
{\color{outcolor}Out[{\color{outcolor}2}]:} <ggplot: (287683557)>
\end{Verbatim}
        
    \begin{Verbatim}[commandchars=\\\{\}]
{\color{incolor}In [{\color{incolor}3}]:} \PY{c+c1}{\PYZsh{} remove grey background}
        \PY{n}{a}\PY{o}{+}\PY{n}{geom\PYZus{}point}\PY{p}{(}\PY{p}{)}\PY{o}{+}\PY{n}{coord\PYZus{}cartesian}\PY{p}{(}\PY{p}{)}\PY{o}{+}\PY{n}{theme\PYZus{}bw}\PY{p}{(}\PY{p}{)}
\end{Verbatim}

    \begin{center}
    \adjustimage{max size={0.9\linewidth}{0.9\paperheight}}{Lecture 13 - Implementing A Layered Grammar of Graphics using Python and plotnine_files/Lecture 13 - Implementing A Layered Grammar of Graphics using Python and plotnine_3_0.png}
    \end{center}
    { \hspace*{\fill} \\}
    
            \begin{Verbatim}[commandchars=\\\{\}]
{\color{outcolor}Out[{\color{outcolor}3}]:} <ggplot: (291141281)>
\end{Verbatim}
        
    \begin{Verbatim}[commandchars=\\\{\}]
{\color{incolor}In [{\color{incolor}4}]:} \PY{c+c1}{\PYZsh{} remove grey background and gridlines}
        \PY{n}{a}\PY{o}{+}\PY{n}{geom\PYZus{}point}\PY{p}{(}\PY{p}{)}\PY{o}{+}\PY{n}{coord\PYZus{}cartesian}\PY{p}{(}\PY{p}{)}\PY{o}{+}\PY{n}{theme\PYZus{}classic}\PY{p}{(}\PY{p}{)}
\end{Verbatim}

    \begin{center}
    \adjustimage{max size={0.9\linewidth}{0.9\paperheight}}{Lecture 13 - Implementing A Layered Grammar of Graphics using Python and plotnine_files/Lecture 13 - Implementing A Layered Grammar of Graphics using Python and plotnine_4_0.png}
    \end{center}
    { \hspace*{\fill} \\}
    
            \begin{Verbatim}[commandchars=\\\{\}]
{\color{outcolor}Out[{\color{outcolor}4}]:} <ggplot: (275809445)>
\end{Verbatim}
        
    \begin{Verbatim}[commandchars=\\\{\}]
{\color{incolor}In [{\color{incolor}5}]:} \PY{c+c1}{\PYZsh{} change the x and y labels; cartesian coordinates are default}
        \PY{n}{a}\PY{o}{+}\PY{n}{geom\PYZus{}point}\PY{p}{(}\PY{p}{)}\PY{o}{+}\PY{n}{theme\PYZus{}classic}\PY{p}{(}\PY{p}{)}\PY{o}{+}\PY{n}{xlab}\PY{p}{(}\PY{l+s+s2}{\PYZdq{}}\PY{l+s+s2}{displacment (l)}\PY{l+s+s2}{\PYZdq{}}\PY{p}{)}\PY{o}{+}\PY{n}{ylab}\PY{p}{(}\PY{l+s+s2}{\PYZdq{}}\PY{l+s+s2}{miles per gallon\PYZhy{}city}\PY{l+s+s2}{\PYZdq{}}\PY{p}{)}
\end{Verbatim}

    \begin{center}
    \adjustimage{max size={0.9\linewidth}{0.9\paperheight}}{Lecture 13 - Implementing A Layered Grammar of Graphics using Python and plotnine_files/Lecture 13 - Implementing A Layered Grammar of Graphics using Python and plotnine_5_0.png}
    \end{center}
    { \hspace*{\fill} \\}
    
            \begin{Verbatim}[commandchars=\\\{\}]
{\color{outcolor}Out[{\color{outcolor}5}]:} <ggplot: (291245617)>
\end{Verbatim}
        
    \begin{Verbatim}[commandchars=\\\{\}]
{\color{incolor}In [{\color{incolor}6}]:} \PY{c+c1}{\PYZsh{} arguments to geom\PYZus{}point() can alter the appearance of the points}
        \PY{n}{a}\PY{o}{=}\PY{n}{ggplot}\PY{p}{(}\PY{n}{mpg}\PY{p}{,}\PY{n}{aes}\PY{p}{(}\PY{n}{x}\PY{o}{=}\PY{l+s+s2}{\PYZdq{}}\PY{l+s+s2}{displ}\PY{l+s+s2}{\PYZdq{}}\PY{p}{,}\PY{n}{y}\PY{o}{=}\PY{l+s+s2}{\PYZdq{}}\PY{l+s+s2}{cty}\PY{l+s+s2}{\PYZdq{}}\PY{p}{)}\PY{p}{)}\PY{o}{+}\PY{n}{geom\PYZus{}point}\PY{p}{(}\PY{n}{color}\PY{o}{=}\PY{l+s+s2}{\PYZdq{}}\PY{l+s+s2}{blue}\PY{l+s+s2}{\PYZdq{}}\PY{p}{,}\PY{n}{shape}\PY{o}{=}\PY{l+m+mi}{4}\PY{p}{,}\PY{n}{size}\PY{o}{=}\PY{l+m+mi}{3}\PY{p}{)}
        \PY{n}{a}\PY{o}{+}\PY{n}{theme\PYZus{}classic}\PY{p}{(}\PY{p}{)}\PY{o}{+}\PY{n}{xlab}\PY{p}{(}\PY{l+s+s2}{\PYZdq{}}\PY{l+s+s2}{displacment (l)}\PY{l+s+s2}{\PYZdq{}}\PY{p}{)}\PY{o}{+}\PY{n}{ylab}\PY{p}{(}\PY{l+s+s2}{\PYZdq{}}\PY{l+s+s2}{miles per gallon\PYZhy{}city}\PY{l+s+s2}{\PYZdq{}}\PY{p}{)}
\end{Verbatim}

    \begin{center}
    \adjustimage{max size={0.9\linewidth}{0.9\paperheight}}{Lecture 13 - Implementing A Layered Grammar of Graphics using Python and plotnine_files/Lecture 13 - Implementing A Layered Grammar of Graphics using Python and plotnine_6_0.png}
    \end{center}
    { \hspace*{\fill} \\}
    
            \begin{Verbatim}[commandchars=\\\{\}]
{\color{outcolor}Out[{\color{outcolor}6}]:} <ggplot: (291474453)>
\end{Verbatim}
        
    \begin{Verbatim}[commandchars=\\\{\}]
{\color{incolor}In [{\color{incolor}7}]:} \PY{c+c1}{\PYZsh{} log transform the y\PYZhy{}axis}
        \PY{n}{a}\PY{o}{=}\PY{n}{ggplot}\PY{p}{(}\PY{n}{mpg}\PY{p}{,}\PY{n}{aes}\PY{p}{(}\PY{n}{x}\PY{o}{=}\PY{l+s+s2}{\PYZdq{}}\PY{l+s+s2}{displ}\PY{l+s+s2}{\PYZdq{}}\PY{p}{,}\PY{n}{y}\PY{o}{=}\PY{l+s+s2}{\PYZdq{}}\PY{l+s+s2}{cty}\PY{l+s+s2}{\PYZdq{}}\PY{p}{)}\PY{p}{)}\PY{o}{+}\PY{n}{geom\PYZus{}point}\PY{p}{(}\PY{p}{)}\PY{o}{+}\PY{n}{theme\PYZus{}classic}\PY{p}{(}\PY{p}{)}
        \PY{n}{a}\PY{o}{+}\PY{n}{xlab}\PY{p}{(}\PY{l+s+s2}{\PYZdq{}}\PY{l+s+s2}{displacment (l)}\PY{l+s+s2}{\PYZdq{}}\PY{p}{)}\PY{o}{+}\PY{n}{ylab}\PY{p}{(}\PY{l+s+s2}{\PYZdq{}}\PY{l+s+s2}{miles per gallon\PYZhy{}city}\PY{l+s+s2}{\PYZdq{}}\PY{p}{)}\PY{o}{+}\PY{n}{scale\PYZus{}y\PYZus{}log10}\PY{p}{(}\PY{p}{)}
\end{Verbatim}

    \begin{center}
    \adjustimage{max size={0.9\linewidth}{0.9\paperheight}}{Lecture 13 - Implementing A Layered Grammar of Graphics using Python and plotnine_files/Lecture 13 - Implementing A Layered Grammar of Graphics using Python and plotnine_7_0.png}
    \end{center}
    { \hspace*{\fill} \\}
    
            \begin{Verbatim}[commandchars=\\\{\}]
{\color{outcolor}Out[{\color{outcolor}7}]:} <ggplot: (291498433)>
\end{Verbatim}
        
    \begin{Verbatim}[commandchars=\\\{\}]
{\color{incolor}In [{\color{incolor}8}]:} \PY{c+c1}{\PYZsh{} arguments to scale can also customize the range and tick locations}
        \PY{n}{a}\PY{o}{=}\PY{n}{ggplot}\PY{p}{(}\PY{n}{mpg}\PY{p}{,}\PY{n}{aes}\PY{p}{(}\PY{n}{x}\PY{o}{=}\PY{l+s+s2}{\PYZdq{}}\PY{l+s+s2}{displ}\PY{l+s+s2}{\PYZdq{}}\PY{p}{,}\PY{n}{y}\PY{o}{=}\PY{l+s+s2}{\PYZdq{}}\PY{l+s+s2}{cty}\PY{l+s+s2}{\PYZdq{}}\PY{p}{)}\PY{p}{)}\PY{o}{+}\PY{n}{geom\PYZus{}point}\PY{p}{(}\PY{p}{)}\PY{o}{+}\PY{n}{theme\PYZus{}classic}\PY{p}{(}\PY{p}{)}
        \PY{n}{a}\PY{o}{+}\PY{n}{scale\PYZus{}y\PYZus{}log10}\PY{p}{(}\PY{n}{limits}\PY{o}{=}\PY{p}{[}\PY{l+m+mi}{1}\PY{p}{,}\PY{l+m+mi}{100}\PY{p}{]}\PY{p}{,}\PY{n}{breaks}\PY{o}{=}\PY{p}{[}\PY{l+m+mi}{1}\PY{p}{,}\PY{l+m+mi}{10}\PY{p}{,}\PY{l+m+mi}{100}\PY{p}{]}\PY{p}{)}
\end{Verbatim}

    \begin{center}
    \adjustimage{max size={0.9\linewidth}{0.9\paperheight}}{Lecture 13 - Implementing A Layered Grammar of Graphics using Python and plotnine_files/Lecture 13 - Implementing A Layered Grammar of Graphics using Python and plotnine_8_0.png}
    \end{center}
    { \hspace*{\fill} \\}
    
            \begin{Verbatim}[commandchars=\\\{\}]
{\color{outcolor}Out[{\color{outcolor}8}]:} <ggplot: (287696481)>
\end{Verbatim}
        
    \begin{Verbatim}[commandchars=\\\{\}]
{\color{incolor}In [{\color{incolor}9}]:} \PY{c+c1}{\PYZsh{} we can also color code points based on continuous or categorical variables}
        \PY{c+c1}{\PYZsh{} continuous}
        \PY{n}{a}\PY{o}{=}\PY{n}{ggplot}\PY{p}{(}\PY{n}{mpg}\PY{p}{,}\PY{n}{aes}\PY{p}{(}\PY{n}{x}\PY{o}{=}\PY{l+s+s2}{\PYZdq{}}\PY{l+s+s2}{displ}\PY{l+s+s2}{\PYZdq{}}\PY{p}{,}\PY{n}{y}\PY{o}{=}\PY{l+s+s2}{\PYZdq{}}\PY{l+s+s2}{cty}\PY{l+s+s2}{\PYZdq{}}\PY{p}{)}\PY{p}{)}\PY{o}{+}\PY{n}{theme\PYZus{}classic}\PY{p}{(}\PY{p}{)}
        \PY{n}{a}\PY{o}{+}\PY{n}{geom\PYZus{}point}\PY{p}{(}\PY{n}{aes}\PY{p}{(}\PY{n}{color}\PY{o}{=}\PY{l+s+s2}{\PYZdq{}}\PY{l+s+s2}{cyl}\PY{l+s+s2}{\PYZdq{}}\PY{p}{)}\PY{p}{)}\PY{o}{+}\PY{n}{xlab}\PY{p}{(}\PY{l+s+s2}{\PYZdq{}}\PY{l+s+s2}{discplacement (l)}\PY{l+s+s2}{\PYZdq{}}\PY{p}{)}\PY{o}{+}\PY{n}{ylab}\PY{p}{(}\PY{l+s+s2}{\PYZdq{}}\PY{l+s+s2}{mpg\PYZhy{}city}\PY{l+s+s2}{\PYZdq{}}\PY{p}{)}
\end{Verbatim}

    \begin{center}
    \adjustimage{max size={0.9\linewidth}{0.9\paperheight}}{Lecture 13 - Implementing A Layered Grammar of Graphics using Python and plotnine_files/Lecture 13 - Implementing A Layered Grammar of Graphics using Python and plotnine_9_0.png}
    \end{center}
    { \hspace*{\fill} \\}
    
            \begin{Verbatim}[commandchars=\\\{\}]
{\color{outcolor}Out[{\color{outcolor}9}]:} <ggplot: (291614565)>
\end{Verbatim}
        
    \begin{Verbatim}[commandchars=\\\{\}]
{\color{incolor}In [{\color{incolor}10}]:} \PY{c+c1}{\PYZsh{} categorical}
         \PY{n}{a}\PY{o}{+}\PY{n}{geom\PYZus{}point}\PY{p}{(}\PY{n}{aes}\PY{p}{(}\PY{n}{color}\PY{o}{=}\PY{l+s+s2}{\PYZdq{}}\PY{l+s+s2}{factor(cyl)}\PY{l+s+s2}{\PYZdq{}}\PY{p}{)}\PY{p}{)}\PY{o}{+}\PY{n}{xlab}\PY{p}{(}\PY{l+s+s2}{\PYZdq{}}\PY{l+s+s2}{discplacement (l)}\PY{l+s+s2}{\PYZdq{}}\PY{p}{)}\PY{o}{+}\PY{n}{ylab}\PY{p}{(}\PY{l+s+s2}{\PYZdq{}}\PY{l+s+s2}{mpg\PYZhy{}city}\PY{l+s+s2}{\PYZdq{}}\PY{p}{)}
\end{Verbatim}

    \begin{center}
    \adjustimage{max size={0.9\linewidth}{0.9\paperheight}}{Lecture 13 - Implementing A Layered Grammar of Graphics using Python and plotnine_files/Lecture 13 - Implementing A Layered Grammar of Graphics using Python and plotnine_10_0.png}
    \end{center}
    { \hspace*{\fill} \\}
    
            \begin{Verbatim}[commandchars=\\\{\}]
{\color{outcolor}Out[{\color{outcolor}10}]:} <ggplot: (291505545)>
\end{Verbatim}
        
    \begin{Verbatim}[commandchars=\\\{\}]
{\color{incolor}In [{\color{incolor}11}]:} \PY{c+c1}{\PYZsh{} categorical \PYZhy{} the default colors are a bit odd}
         \PY{n}{a}\PY{o}{=}\PY{n}{ggplot}\PY{p}{(}\PY{n}{mpg}\PY{p}{,}\PY{n}{aes}\PY{p}{(}\PY{n}{x}\PY{o}{=}\PY{l+s+s2}{\PYZdq{}}\PY{l+s+s2}{displ}\PY{l+s+s2}{\PYZdq{}}\PY{p}{,}\PY{n}{y}\PY{o}{=}\PY{l+s+s2}{\PYZdq{}}\PY{l+s+s2}{cty}\PY{l+s+s2}{\PYZdq{}}\PY{p}{)}\PY{p}{)}\PY{o}{+}\PY{n}{geom\PYZus{}point}\PY{p}{(}\PY{n}{aes}\PY{p}{(}\PY{n}{color}\PY{o}{=}\PY{l+s+s2}{\PYZdq{}}\PY{l+s+s2}{factor(cyl)}\PY{l+s+s2}{\PYZdq{}}\PY{p}{)}\PY{p}{)}
         \PY{n}{a}\PY{o}{+}\PY{n}{scale\PYZus{}color\PYZus{}manual}\PY{p}{(}\PY{n}{values}\PY{o}{=}\PY{p}{[}\PY{l+s+s1}{\PYZsq{}}\PY{l+s+s1}{red}\PY{l+s+s1}{\PYZsq{}}\PY{p}{,}\PY{l+s+s1}{\PYZsq{}}\PY{l+s+s1}{green}\PY{l+s+s1}{\PYZsq{}}\PY{p}{,}\PY{l+s+s1}{\PYZsq{}}\PY{l+s+s1}{blue}\PY{l+s+s1}{\PYZsq{}}\PY{p}{,}\PY{l+s+s1}{\PYZsq{}}\PY{l+s+s1}{orange}\PY{l+s+s1}{\PYZsq{}}\PY{p}{]}\PY{p}{)}\PY{o}{+}\PY{n}{theme\PYZus{}classic}\PY{p}{(}\PY{p}{)}
\end{Verbatim}

    \begin{center}
    \adjustimage{max size={0.9\linewidth}{0.9\paperheight}}{Lecture 13 - Implementing A Layered Grammar of Graphics using Python and plotnine_files/Lecture 13 - Implementing A Layered Grammar of Graphics using Python and plotnine_11_0.png}
    \end{center}
    { \hspace*{\fill} \\}
    
            \begin{Verbatim}[commandchars=\\\{\}]
{\color{outcolor}Out[{\color{outcolor}11}]:} <ggplot: (291616897)>
\end{Verbatim}
        
    \begin{Verbatim}[commandchars=\\\{\}]
{\color{incolor}In [{\color{incolor}12}]:} \PY{c+c1}{\PYZsh{} add a linear trendline with a new layer}
         \PY{n}{a}\PY{o}{=}\PY{n}{ggplot}\PY{p}{(}\PY{n}{mpg}\PY{p}{,}\PY{n}{aes}\PY{p}{(}\PY{n}{x}\PY{o}{=}\PY{l+s+s2}{\PYZdq{}}\PY{l+s+s2}{displ}\PY{l+s+s2}{\PYZdq{}}\PY{p}{,}\PY{n}{y}\PY{o}{=}\PY{l+s+s2}{\PYZdq{}}\PY{l+s+s2}{cty}\PY{l+s+s2}{\PYZdq{}}\PY{p}{)}\PY{p}{)}\PY{o}{+}\PY{n}{theme\PYZus{}classic}\PY{p}{(}\PY{p}{)}\PY{o}{+}\PY{n}{geom\PYZus{}point}\PY{p}{(}\PY{p}{)}
         \PY{n}{a}\PY{o}{+}\PY{n}{xlab}\PY{p}{(}\PY{l+s+s2}{\PYZdq{}}\PY{l+s+s2}{displacment (l)}\PY{l+s+s2}{\PYZdq{}}\PY{p}{)}\PY{o}{+}\PY{n}{ylab}\PY{p}{(}\PY{l+s+s2}{\PYZdq{}}\PY{l+s+s2}{miles per gallon\PYZhy{}city}\PY{l+s+s2}{\PYZdq{}}\PY{p}{)}\PY{o}{+}\PY{n}{stat\PYZus{}smooth}\PY{p}{(}\PY{n}{method}\PY{o}{=}\PY{l+s+s2}{\PYZdq{}}\PY{l+s+s2}{lm}\PY{l+s+s2}{\PYZdq{}}\PY{p}{)}
\end{Verbatim}

    \begin{center}
    \adjustimage{max size={0.9\linewidth}{0.9\paperheight}}{Lecture 13 - Implementing A Layered Grammar of Graphics using Python and plotnine_files/Lecture 13 - Implementing A Layered Grammar of Graphics using Python and plotnine_12_0.png}
    \end{center}
    { \hspace*{\fill} \\}
    
            \begin{Verbatim}[commandchars=\\\{\}]
{\color{outcolor}Out[{\color{outcolor}12}]:} <ggplot: (292024209)>
\end{Verbatim}
        
    \begin{Verbatim}[commandchars=\\\{\}]
{\color{incolor}In [{\color{incolor}13}]:} \PY{c+c1}{\PYZsh{} add a spline with a new layer}
         \PY{n}{a}\PY{o}{+}\PY{n}{xlab}\PY{p}{(}\PY{l+s+s2}{\PYZdq{}}\PY{l+s+s2}{displacment (l)}\PY{l+s+s2}{\PYZdq{}}\PY{p}{)}\PY{o}{+}\PY{n}{ylab}\PY{p}{(}\PY{l+s+s2}{\PYZdq{}}\PY{l+s+s2}{miles per gallon\PYZhy{}city}\PY{l+s+s2}{\PYZdq{}}\PY{p}{)}\PY{o}{+}\PY{n}{stat\PYZus{}smooth}\PY{p}{(}\PY{n}{method}\PY{o}{=}\PY{l+s+s2}{\PYZdq{}}\PY{l+s+s2}{loess}\PY{l+s+s2}{\PYZdq{}}\PY{p}{)}
\end{Verbatim}

    \begin{center}
    \adjustimage{max size={0.9\linewidth}{0.9\paperheight}}{Lecture 13 - Implementing A Layered Grammar of Graphics using Python and plotnine_files/Lecture 13 - Implementing A Layered Grammar of Graphics using Python and plotnine_13_0.png}
    \end{center}
    { \hspace*{\fill} \\}
    
            \begin{Verbatim}[commandchars=\\\{\}]
{\color{outcolor}Out[{\color{outcolor}13}]:} <ggplot: (291599101)>
\end{Verbatim}
        
    \begin{Verbatim}[commandchars=\\\{\}]
{\color{incolor}In [{\color{incolor}14}]:} \PY{c+c1}{\PYZsh{} histogram of mpg hwy}
         \PY{n}{b}\PY{o}{=}\PY{n}{ggplot}\PY{p}{(}\PY{n}{mpg}\PY{p}{,}\PY{n}{aes}\PY{p}{(}\PY{n}{x}\PY{o}{=}\PY{l+s+s2}{\PYZdq{}}\PY{l+s+s2}{hwy}\PY{l+s+s2}{\PYZdq{}}\PY{p}{)}\PY{p}{)}
         \PY{n}{b}\PY{o}{+}\PY{n}{geom\PYZus{}histogram}\PY{p}{(}\PY{p}{)}\PY{o}{+}\PY{n}{theme\PYZus{}classic}\PY{p}{(}\PY{p}{)}
\end{Verbatim}

    \begin{Verbatim}[commandchars=\\\{\}]
/anaconda/lib/python2.7/site-packages/plotnine/stats/stat\_bin.py:90: UserWarning: 'stat\_bin()' using 'bins = 11'. Pick better value with 'binwidth'.
  warn(msg.format(params['bins']))

    \end{Verbatim}

    \begin{center}
    \adjustimage{max size={0.9\linewidth}{0.9\paperheight}}{Lecture 13 - Implementing A Layered Grammar of Graphics using Python and plotnine_files/Lecture 13 - Implementing A Layered Grammar of Graphics using Python and plotnine_14_1.png}
    \end{center}
    { \hspace*{\fill} \\}
    
            \begin{Verbatim}[commandchars=\\\{\}]
{\color{outcolor}Out[{\color{outcolor}14}]:} <ggplot: (292284833)>
\end{Verbatim}
        
    \begin{Verbatim}[commandchars=\\\{\}]
{\color{incolor}In [{\color{incolor}15}]:} \PY{c+c1}{\PYZsh{} change bins}
         \PY{n}{b}\PY{o}{+}\PY{n}{geom\PYZus{}histogram}\PY{p}{(}\PY{n}{binwidth}\PY{o}{=}\PY{l+m+mi}{5}\PY{p}{)}\PY{o}{+}\PY{n}{theme\PYZus{}classic}\PY{p}{(}\PY{p}{)}
\end{Verbatim}

    \begin{center}
    \adjustimage{max size={0.9\linewidth}{0.9\paperheight}}{Lecture 13 - Implementing A Layered Grammar of Graphics using Python and plotnine_files/Lecture 13 - Implementing A Layered Grammar of Graphics using Python and plotnine_15_0.png}
    \end{center}
    { \hspace*{\fill} \\}
    
            \begin{Verbatim}[commandchars=\\\{\}]
{\color{outcolor}Out[{\color{outcolor}15}]:} <ggplot: (291266925)>
\end{Verbatim}
        
    \begin{Verbatim}[commandchars=\\\{\}]
{\color{incolor}In [{\color{incolor}16}]:} \PY{c+c1}{\PYZsh{} color can also be specified in the geom\PYZus{}histogram call}
         \PY{n}{b}\PY{o}{+}\PY{n}{geom\PYZus{}histogram}\PY{p}{(}\PY{n}{binwidth}\PY{o}{=}\PY{l+m+mi}{5}\PY{p}{,}\PY{n}{fill}\PY{o}{=}\PY{l+s+s1}{\PYZsq{}}\PY{l+s+s1}{red}\PY{l+s+s1}{\PYZsq{}}\PY{p}{,}\PY{n}{color}\PY{o}{=}\PY{l+s+s1}{\PYZsq{}}\PY{l+s+s1}{black}\PY{l+s+s1}{\PYZsq{}}\PY{p}{)}\PY{o}{+}\PY{n}{theme\PYZus{}classic}\PY{p}{(}\PY{p}{)}
\end{Verbatim}

    \begin{center}
    \adjustimage{max size={0.9\linewidth}{0.9\paperheight}}{Lecture 13 - Implementing A Layered Grammar of Graphics using Python and plotnine_files/Lecture 13 - Implementing A Layered Grammar of Graphics using Python and plotnine_16_0.png}
    \end{center}
    { \hspace*{\fill} \\}
    
            \begin{Verbatim}[commandchars=\\\{\}]
{\color{outcolor}Out[{\color{outcolor}16}]:} <ggplot: (291717901)>
\end{Verbatim}
        
    \begin{Verbatim}[commandchars=\\\{\}]
{\color{incolor}In [{\color{incolor}17}]:} \PY{c+c1}{\PYZsh{} we can generate a barplot of means too}
         \PY{n}{d}\PY{o}{=}\PY{n}{ggplot}\PY{p}{(}\PY{n}{mpg}\PY{p}{)}\PY{o}{+}\PY{n}{theme\PYZus{}classic}\PY{p}{(}\PY{p}{)}\PY{o}{+}\PY{n}{xlab}\PY{p}{(}\PY{l+s+s2}{\PYZdq{}}\PY{l+s+s2}{drive}\PY{l+s+s2}{\PYZdq{}}\PY{p}{)}\PY{o}{+}\PY{n}{ylab}\PY{p}{(}\PY{l+s+s2}{\PYZdq{}}\PY{l+s+s2}{miles per gallon city}\PY{l+s+s2}{\PYZdq{}}\PY{p}{)}
         \PY{n}{d}\PY{o}{+}\PY{n}{geom\PYZus{}bar}\PY{p}{(}\PY{n}{aes}\PY{p}{(}\PY{n}{x}\PY{o}{=}\PY{l+s+s2}{\PYZdq{}}\PY{l+s+s2}{factor(drv)}\PY{l+s+s2}{\PYZdq{}}\PY{p}{,}\PY{n}{y}\PY{o}{=}\PY{l+s+s2}{\PYZdq{}}\PY{l+s+s2}{cty}\PY{l+s+s2}{\PYZdq{}}\PY{p}{)}\PY{p}{,}\PY{n}{stat}\PY{o}{=}\PY{l+s+s2}{\PYZdq{}}\PY{l+s+s2}{summary}\PY{l+s+s2}{\PYZdq{}}\PY{p}{,}\PY{n}{fun\PYZus{}y}\PY{o}{=}\PY{n}{numpy}\PY{o}{.}\PY{n}{mean}\PY{p}{)}
\end{Verbatim}

    \begin{center}
    \adjustimage{max size={0.9\linewidth}{0.9\paperheight}}{Lecture 13 - Implementing A Layered Grammar of Graphics using Python and plotnine_files/Lecture 13 - Implementing A Layered Grammar of Graphics using Python and plotnine_17_0.png}
    \end{center}
    { \hspace*{\fill} \\}
    
            \begin{Verbatim}[commandchars=\\\{\}]
{\color{outcolor}Out[{\color{outcolor}17}]:} <ggplot: (291344681)>
\end{Verbatim}
        
    \begin{Verbatim}[commandchars=\\\{\}]
{\color{incolor}In [{\color{incolor}18}]:} \PY{c+c1}{\PYZsh{} calculate means by drv to check barplot}
         \PY{n}{mpg}\PY{o}{.}\PY{n}{groupby}\PY{p}{(}\PY{p}{[}\PY{l+s+s1}{\PYZsq{}}\PY{l+s+s1}{drv}\PY{l+s+s1}{\PYZsq{}}\PY{p}{]}\PY{p}{)}\PY{p}{[}\PY{l+s+s1}{\PYZsq{}}\PY{l+s+s1}{cty}\PY{l+s+s1}{\PYZsq{}}\PY{p}{]}\PY{o}{.}\PY{n}{mean}\PY{p}{(}\PY{p}{)}
\end{Verbatim}

            \begin{Verbatim}[commandchars=\\\{\}]
{\color{outcolor}Out[{\color{outcolor}18}]:} drv
         4    14.330097
         f    19.971698
         r    14.080000
         Name: cty, dtype: float64
\end{Verbatim}
        
    \begin{Verbatim}[commandchars=\\\{\}]
{\color{incolor}In [{\color{incolor}19}]:} \PY{c+c1}{\PYZsh{} faceting allows the same plot for different classes to be generated}
         \PY{n}{f}\PY{o}{=}\PY{n}{ggplot}\PY{p}{(}\PY{n}{mpg}\PY{p}{,}\PY{n}{aes}\PY{p}{(}\PY{n}{x}\PY{o}{=}\PY{l+s+s2}{\PYZdq{}}\PY{l+s+s2}{displ}\PY{l+s+s2}{\PYZdq{}}\PY{p}{,}\PY{n}{y}\PY{o}{=}\PY{l+s+s2}{\PYZdq{}}\PY{l+s+s2}{cty}\PY{l+s+s2}{\PYZdq{}}\PY{p}{)}\PY{p}{)}\PY{o}{+}\PY{n}{geom\PYZus{}point}\PY{p}{(}\PY{p}{)}
         \PY{n}{f}\PY{o}{+}\PY{n}{facet\PYZus{}wrap}\PY{p}{(}\PY{l+s+s2}{\PYZdq{}}\PY{l+s+s2}{\PYZti{}drv}\PY{l+s+s2}{\PYZdq{}}\PY{p}{)}\PY{o}{+}\PY{n}{theme\PYZus{}classic}\PY{p}{(}\PY{p}{)}
\end{Verbatim}

    \begin{center}
    \adjustimage{max size={0.9\linewidth}{0.9\paperheight}}{Lecture 13 - Implementing A Layered Grammar of Graphics using Python and plotnine_files/Lecture 13 - Implementing A Layered Grammar of Graphics using Python and plotnine_19_0.png}
    \end{center}
    { \hspace*{\fill} \\}
    
            \begin{Verbatim}[commandchars=\\\{\}]
{\color{outcolor}Out[{\color{outcolor}19}]:} <ggplot: (291275085)>
\end{Verbatim}
        
    \subsection*{Challenge}\label{challenge}

Practice using the syntax demonstrated above by writing a script to
generate the following plots using the mpg data.

1. A scatter plot of miles per gallon city versus miles per gallon
highway. Color code the points by 'drv' (four-wheel drive vs.
front-wheel drive vs. rear-wheel drive). Add a linear trendline to the
plot.

~

2. A "density plot" of engine displacement.

~

3. A barplot of mean displacement for different numbers of cylinders
(cyl).


    % Add a bibliography block to the postdoc
    
    
    
    \end{document}
